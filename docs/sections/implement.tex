\section{Implementation}
\label{sec:implement}

The interpreter for the Gryph Programming Language,
implemented in Haskell language during this semester, comprises three
main steps: \textbf{lexical analysis}, for tokens discovery;
\textbf{syntactic analysis}, for language constructs
identification; and \textbf{program execution}, for
performing machine state changes according to the
meaning of the parsed entities, given by the semantic
rules of the language (see Section \ref{sec:usage}). The following subsections
present them with rich detail.

\subsection{Lexical analysis}

Lexical analysis is the part of the task of analysing syntax
which deals with small-scale constructs. Dealing with it
separately is justified by gains in terms of simplicity, since
it uses simple techniques (pattern matching, essentially); efficiency, since it is
more suitable for optimization than the general syntactic parser; 
and portability, since lexical analysers can be platform dependent, in contrast
with syntax analysers, which can be made independent.

In order to simplify the development, \textbf{Alex}, a lexical
analyzer generator for Haskell, was used. It is basically built
upon a sequence of regular expressions which describes the 
language tokens. The implemented lexical analyzer takes a program as input and produces
a list of the identified tokens, which is then transmitted to the
syntactic parser, described in the next subsection. Alex also furnishes
the position (line and column) of each token, for further richier error reporting, but
this version of the interpreter does not take it into account.

\subsection{Syntactic analysis}


\subsection{Program execution}
