\section{Introduction}
\label{sec:intro}
Gryph is a programming language designed and developed for graph-oriented programming. It is easy to learn and to program with, being a relatively abstract scripting language, with many built-in high-level data structures. Gryph's static typing, along with its interpreted nature, makes it ideal for prototyping fast and securely. But most of all, it is the ease of constructing and manipulating graphs it offers that sets it apart from other programming languages.

One domain in which Gryph may perform best is in the teaching and learning of common graph algorithms, where constructing the necessary data structures may get in the way of implementing such algorithms, and of observing exactly how they work. And yet, it is not limited to a learning context, being powerful enough to be used for implementing solutions to most problems of non-critical performance demands.

This document covers, firstly, the basic functionalities of the language (\ref{sec:manual}), in terms of both syntax and semantics; then it expounds on the first, and so far only, implementation of Gryph (\ref{sec:implement}); and finally it presents the Gryph syntax in EBNF (\ref{}). Being an actively developed project, parts of this document may be inaccurate or outdated: in case you detect any inconsistencies, we ask that you contact the authors of this document or raise an issue in the official Gryph repository.