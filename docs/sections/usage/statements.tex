\subsection{Statements}
\label{sec:statements}
A statement is the basic building block of a Gryph program. It comes in two forms: simple statements (\ref{sec:simple-stmts}); or compound statements (\ref{sec:compound-stmts}), which may contain other statements, compound or simple.

\subsubsection{Simple statements}
\label{sec:simple-stmts}
A simple statement can be: a variable declaration or assignment (\ref{sec:variables}), an I/O statement (\ref{sec:io}), an insertion or removal operation over a composite type (\ref{sec:add-del}), or a return (\ref{sec:return}) or escape (\ref{sec:escape}) statement. The end of a simple statement is always marked by a semicolon.

\paragraph{Variable declaration and assignment}
\label{sec:variables}

A variable declaration statement consists of an identifier, or a list of comma-separated identifiers, followed by a colon and a valid type. This declares a variable of the given type for each of the identifiers used. Optionally, the declared variables can be initialized, with an equals sign followed by an expression or a list of comma-separated expressions, after the declared type. 

If a list of identifiers is initialized with a single expression, that expression is assigned to all the declared variables, but if they are initialized with a list of expressions, each expression is assigned respectively to each variable corresponding to that identifier. A declaration with a list of identifiers and a list of initializing expressions of different sizes is not meaningful, and is in fact invalid.

A variable assignment consists of an expression, or list of comma-separated expressions, where each expression must evaluate to a variable, followed by an equals sign and a single expression or a list of expressions. Assignment to multiple variables works in the same way as initialization of multiple variables.

The following code declares and initializes three integers, and reassigns values to two of them. The printed values are shown as comments.
\begin{lstlisting}[language=Gryph]
a, b, c : int = 0;
a, c = -1, 1;
println a; # -1
println b; # 0
println c; # 1
\end{lstlisting}

\paragraph{I/O statements}
\label{sec:io}

As for I/O output statements, there is \key{read} for input and \key{print} or \key{println} for output. For input, the \key{read} keyword is used, followed by a variable of the type \type{string}, where \key{read} will store the one line it reads from standard input. Numeric values can be ``read'' by attempting to cast the read string to the desired numeric type (see \ref{sec:casting}).

And for output, the \key{print} statement, followed by any expression, will print a string representing the value of that expression to standard output. The \key{println} alternative works in the same way as \key{print}, put also prints a line break after the printed value.

The following code simply reads a line from standard input an prints it back.
\begin{lstlisting}[language=Gryph]
s : string;
read s;
println s;
\end{lstlisting}

\paragraph{Insertion and removal statements}
\label{sec:add-del}
Now, in regards to insertion and removal, the two relevant keywords to be used are \key{add} and \key{del}, respectively. Lists support both insertion and removal: to insert a new element \id{x} to a list \id{y}, the correct syntax is \key{add} \id{x} \key{in} \id{y}, which appends \id{x} to the end of \id{y}, increasing the list's size by one. To remove an element from a list, one might write \key{del} \id{p} \key{from} \id{y}, where \id{p} is an integer indicating the position in the list of the element to be removed.

The \type{tuple}, by contrast, supports neither insertion nor removal, being immutable; while the \type{dictionary} type supports removal, but insertion only through assignment to a key not yet in it. For removal of an entry with key \id{k} from a dictionary \id{d}, the expected syntax is \key{del} \id{k} \key{from} \id{d}. Note that just as a list will not allow an element to be removed from an invalid position, a dictionary will not allow removal of an entry with a certain key if it contains no such entry.

The \type{graph} type supports insertion and removal of both vertices and edges. To insert a vertex \id{v} in a graph \id{g}: \key{add} \id{v} \key{in} \id{g}. To remove it: \key{del} \id{v} \key{from} \id{g}. Removing a vertex from a graph also removes all edges connected to it. Inserting and removing edges works in a similar way, but with the option of using the \key{where} keyword to specify a value to the edge being inserted, as in graph comprehension (see \ref{sec:comprehension}).

In the following code we exemplify some of the statements discussed in this section.
\begin{lstlisting}[language=Gryph]
l : [int] = [-5, 0];
add 5 in l;
del 1 from l;
println l; # [-5, 5]

d : |string, bool| = |"white" ? false|;
d|"black"| = true;
del "white" from d;
println d; # |"black" ? true|

g : <int, char> = <[1]>;
add 2 in g;
add 'x' where 1->2 in g;
println g<1->2>; # 'x'
del 1->2 from g;
\end{lstlisting}

\paragraph{Return statement}
\label{sec:return}
The return statement is one that may only be used inside subprograms, it ends the function immediately, executing none of the statements that would follow it, and returns a value if necessary. For functions, in particular, the \key{return} keyword is always followed by an expression, which must be evaluated to the same type as that function's return type. For procedures (subprograms with no return value), the return statement consists of only the \key{return} keyword.

\paragraph{Escape statement}
\label{sec:escape}
An escape statement is used to exit an iteration structure (see \ref{sec:iteration}) independently to its regular exit conditions, continuing execution from the first statement following the end of the iteration structure wherein it was called. As such, it may only be used inside an iteration structure, be it a \key{for} or a \key{while}. Note that this escape statement is not ``multi-level'', that is, when called from within a nested loop, it exits only the innermost loop.

\subsubsection{Compound statements}
\label{sec:compound-stmts}
A compound statement can also be defined as a control structure, in that they alter the sequential flow of control based on certain conditions. They are divided in conditional structures (\ref{sec:conditional}) and iteration structures (\ref{sec:iteration}).

\paragraph{Conditional structures}
\label{sec:conditional}
Conditional structures are constructed with the keywords \key{if} and \key{else}. These work in much the same way as in other programming languages, such as Java. The former, \key{if}, is always followed by an expression enclosed in parentheses, which must evaluate to a boolean value. If it is true, the code in the braces-enclosed block that follows it is executed. The latter, \key{else}, is always preceded by an \key{if}-block. The block that follows \key{else} is executed if and only if the preceding block wasn't. The blocks that follow \key{if} or \key{else} may actually be a single statement, alternatively to one or many statements enclosed by braces.

The following code will print a different message depending on the value of variable \id{a} (which must be of type \type{int} or \type{float}).
\begin{lstlisting}[language=Gryph]
if (a > 0) {
	println "greater";
} else if (a == 0) {
	println "equal";
} else {
	println "lesser";
}
\end{lstlisting}

\paragraph{Iteration}
\label{sec:iteration}
Another important part of control flow statements are the iteration constructs. Gryph has types of these: a `while' loop, which repeats code while a certain condition is true; a `for' loop, which iterate over a composite type, such as a list; and loops for specific traversal of a graph, which can be a BFS loop or a DFS loop.

A while loop is constructed with the keyword \key{while}, followed by an expression enclosed in parentheses which must evaluate to a boolean value, and then followed by a block of statements enclosed in braces: the code that will be repeated in each iteration of the loop. The block is repeated every time it reaches its end and the condition after \key{while} is still true. If the condition is false when execution first reaches the \key{while} statement, the block is never executed.

In the code below, we use a while loop to print multiple values of \id{n}, shown as comments below the loop.
\begin{lstlisting}[language=Gryph]
n : int = 3;
while (n > 0) {
	println n;
	n = n - 1;
}
# 3
# 2
# 1
println n; # 0
\end{lstlisting}

A for loop, commonly called in this case a foreach loop, since it iterates over elements in a data structure, is composed of the keyword \key{for}, followed by an identifier or list of identifiers, the \key{over} keyword, and the expression or list of expressions that will evaluate to values of composite type over which you will iterate. Then comes the block that will be executed in each iteration.

In the case of lists, you might have one identifier in the left of \key{over} and one list in the right: in this case, a variable with the identifier given may be used inside the \key{for}-block, with its value assuming at each iteration the value of the next element in the list on the right side of \key{over}, starting with the first. In the case of multiple identifiers and lists on either side of \key{over}, the iteration works much like a nested loop: for each value of the first list assumed by the first identifier's variable, all values of the second list are assumed in order by the second identifier's variable, and so on.

We exemplify both of these cases in the code below. The first loop should have the same output as the while loop shown above, and the output of the second is shown as comments below it.
\begin{lstlisting}[language=Gryph]
for i over [3, 2, 1] {
	println i;
}

for i, j over [0, 1], [2, 4] {
	println i + j;
}
# 2 
# 4 
# 3 
# 5 
\end{lstlisting}

Iteration over a dictionary works in a similar way to a list, but the type of the iteration variable is that of a (key, value) tuple, for the key and value types of the dictionary being iterated over, to which the values for each of the dictionary's entry is assigned in turn. In the code below, we iterate over a dictionary \id{d}, printing ``true'' for each of its entries.

\begin{lstlisting}[language=Gryph]
d : |int, char| = |2 ? 'x', 1 ? 'y'|;
for entry over d {
	println d|entry\0\| == entry\1\;
}
\end{lstlisting}

As for graph iteration, the \key{for} syntax works in the same way as a list, iterating over the graph's vertices in no particular order. To specify an order, a graph-traversal keyword such as \key{bfs} or \key{dfs} may be used. When iterating through a graph using \key{bfs} or \key{dfs}, up to three identifiers may be specified, which will assume at each iteration the values of, respectively: the current vertex, its `parent' vertex, and the current distance or depth from the starting vertex.

As with conditional structures, the blocks contained in iteration structures may actually be a single statement, alternatively to one or many statements enclosed by braces.